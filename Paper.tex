\documentclass[jou]{apa6}

\usepackage[spanish]{babel}
\usepackage[utf8]{inputenc}
\usepackage{csquotes}
\usepackage{siunitx}
\usepackage[style=apa,sortcites=true,sorting=nyt,backend=biber]{biblatex}

\DeclareLanguageMapping{spanish}{spanish-apa}
\addbibresource{bibliography.bib}

\title{Conectividad cerebral y machine learning. Dos campos de acción, un paradigma en común}
\shorttitle{Conectividad cerebral y machine learning}
\author{Juan D. Angarita y Sharol G. Torres}
\affiliation{Universidad Popular del Cesar}
\date{2018-08-08}

\leftheader{Angarita y Torres}
\rightheader{Conectividad cerebral y machine learning}


\abstract{¿Desde qué punto se interrelaciona la Neurociencia y la Inteligencia Artificial (IA)? 
¿Dónde están los límites de esta fusión? Ante estos interrogantes se darán respuesta a lo largo 
del presente trabajo, realizando para ello un itinerario sobre las teorías neurocientíficas y su 
influencia en el desarrollo de otras ciencias cognitivas como la I.A. Si bien la neurociencia y 
la inteligencia artificial se habían considerado como campos independientes en un principio, 
hoy en día esa delimitación parece obrada de modo artificial y es correspondida con la realidad 
de la investigación sobre el tema en cuestión.
En los últimos tiempos, la Inteligencia Artificial (IA) ha cobrado un importante papel en el intento 
de posibilitar la simulación de las vías ejecutivas (conectividad cerebral) de procesos cognitivos 
como la lectura; es a través de esta tentativa que se ha puesto 
en marcha nuestra investigación.
}

\keywords{machine learning, inteligencia artificial, conectividad cerebral}


\begin{document}
\maketitle

Lo que vemos, lo vemos no sólo en función de la estimulación entrante, sino de
interferencias inconscientes en el procesamiento visual (Helmholtz, 1866). Tal vez la
demostración más poderosa de esto son las imágenes biestables: aquellas en las que el mismo
estímulo se alterna entre dos percepciones muy diferentes, correspondiendo a dos estados 
estables compitiendo en un sistema dinámico subyacente. Ward Scholl, 2015.

Los cambios perceptuales en las imágenes biestables pueden ser activados por múltiples
factores, sin embargo, podemos clasificarlos en categorías: Primero, y tal vez el más intuitivo
de todos es que el cambio puede ocurrir sin ninguna razón en concreto. Esta posibilidad es
apoyada por la fenomenología de cambios no deseados y por la evidencia de que los periodos
de dominio y supresión están caracterizados por una independencia estocástica.

Segundo, los cambios perceptuales pueden ser sistemáticamente influenciados por las
diferentes propiedades intrínsecas del estímulo mismo. Consideremos, por ejemplo, un
estímulo en el que un grupo de puntos moviéndose puede ser visto como una esfera
tridimensional transparente rotando hacia la derecha o la izquierda. Si la fuerza de una de las
interpretaciones es mejorada (simplemente hacer que los puntos se muevan en una dirección
de forma más brillante que las otras) entonces la percepción impulsada tenderá a dominar
. De forma similar, simplemente el hecho de ver una
interpretación por un periodo extendido de tiempo introducirá la adaptación, haciendo que el
cambio a la interpretación contraria sea más probable a medida que el tiempo va escalando.

Tercero, la estimulación visual totalmente extrínseca puede inducir el cambio perceptual.
Por ejemplo, simplemente la presentación de un breve (pero consciente) estímulo visual
puede inducir un subsecuente cambio perceptual mientras se observa un fenómeno como el
cubo de Necker.

Cuarto, la percepción de imágenes biestables puede cambiar debido a un esfuerzo
voluntario, es decir, porque nosotros explícitamente intentamos hacerlas cambiar. Los
observadores pueden mantener una percepción específica o pueden cambiar entre
percepciones basándose en instrucciones explícitas de lo que deben hacer (Toppino, 2003).
Por ejemplo, parece que algunos cambios intencionales en el cubo de Necker se producen
debido a la priorización atencional de algunos contornos y vértices específicos.

El cambio también puede ser influenciado por el conocimiento explícito de la
multiestabilidad misma y las posibles percepciones compitiendo. Ocurre más fácilmente, por
ejemplo, cuando las posibles interpretaciones son explícitamente reconocibles, 
y de forma inversa una percepción que de otra manera sería dominante puede
ver dramáticamente reducida su dominancia perpceptual ante el desconocimiento de la
biestabilidad por parte del observador.

Finalmente, en el 2015 se encontró que los cambios de percepción pueden producirse por
la presentación sistemática de señales inconscientes. Pequeñas alteraciones en forma de
señales, como marcas de oclusión ambiental, podrían disminuir la aleatoriedad de los
cambios de percepción de estímulos biestables.

Nuestro estudio se basó en la exploración de una de las posibilidades implementadas por
Ward y Scholl, quienes en el año 2015 llevaron a cabo el experimento más completo hasta la
fecha en relación a este tema. Utilizaremos un método diferente para la presentación de las
señales inconscientes, enfocándonos en reducir la posibilidad de confundir un cambio
estocástico con uno producido por la señal y observaremos si este cambio tiene alguna
incidencia en los resultados recolectados por Ward y Scholl.

En este punto estamos interesados en una posibilidad que no encaja limpiamente dentro
de ninguna de esas categorías en particular, en la cual, señales inconscientes transitorias
podrían ocasionar cambios perceptuales subsecuentes que no obstante a ojos de los
observadores parecerían ser completamente aleatorios. A diferencia de la mayoría de estudios
de factores de percepción de “arriba hacia abajo” (ej, esfuerzo voluntario o conocimiento de
la ambigüedad), esta posibilidad involucra específicamente estímulos
de percepción de “abajo hacia arriba”. A diferencia de estudios que cambian la naturaleza del
estímulo que será visto (ej., Klink et al., 2008), el presente experimento involucra señales que
no son, por si mismas parte de la percepción resultante. (No es simplemente que el
observador no pueda apreciar como y porque las señales están influenciando su percepción;
sino que, ellos no pueden si quiera ver las señales para empezar). Finalmente, mientras que
hay muchos otros ejemplos de señales no vistas influenciando lo que vemos después, 
aquí nosotros exploramos como una señal transitoria no vista
puede influenciar una percepción estable en curso, incluso cuando ni el contenido ni la
existencia de esa información llegue a ser consciente.

Si dichas señales inconscientes influencian la percepción resultante de una forma
dependiente de su contenido particular, esto indicará que la señal visual (procesamiento de
abajo hacia arriba) puede influenciar el cambio de percepción incluso cuando las señales no
figuran en la percepción resultante, e incluso
cuando los observadores sienten que los cambios son enteramente estocásticos.

Respecto a la animación utilizada, algunos de los investigadores también han apuntado
que mientras la figura es una silueta pura, hay algunas señales sutiles de profundidad en la
animación original (Troje, N. F., 2010). En particular, los observadores tienden a ver las
figuras ambiguas como si estuvieran posicionadas debajo de su punto de vista y esto,
combinado de forma sutil con una sombra visible puede llevar a que la percepción en
dirección de las agujas del reloj sea dominante.

A pesar de estos sesgos y la resistencia inicial al cambio intencional, la bailarina es
biestable y los observadores perciben el cambio. Aunque la
mayoría de los cambios parecen ser puramente estocásticos, el cambio puede ser influenciado
por factores de procesamiento de “abajo hacia arriba” (ej, se producen más cambios cuando
la animación gira más rápido) y de procesamiento de “arriba hacia abajo” (ej,
intencionalmente enfocándose en los pies de la bailarina. Liu et al., 2012).

\section{Métodos}
\subsection{Participantes}
Para la recolección de los datos necesarios, el experimento fue realizado (previa
autorización) en un grupo de 20 individuos (edad promedio = 19 años), de los cuales se
pudieron obtener 19 muestras exitosas (un individuo reportó notar la señal y por lo tanto su
información no fue tenida en cuenta).

\subsection{Materiales}
El estímulo utilizado es la versión original de la ilusión de la bailarina que se puede
obtener desde la página web del autor. Esta animación (\ang{8,85} X \ang{10,00}) representa a una
figura femenina como una silueta negra sobre un fondo de grises degradados (con una sombra
visible debajo de uno de sus pies, vista desde una elevación de \ang{6,8} desde el plano horizontal,
con una extensión vertical máxima de \ang{7,15}, rotando \ang{176,47} por segundo en el plano frontal,
por lo tanto una rotación completa de \ang{360} sucede aproximadamente cada 2 segundos (2,040
ms, cada uno de los 34 frames de la imagen es presentado por 60 ms). Debido a la ausencia
de señales de profundidad, la bailarina puede ser percibida como si estuviera rotando a favor
o en contra de la dirección de las agujas del reloj.

\subsection{Diseño}
Esta animación fue presentada utilizando una aplicación móvil para el sistema operativo
Android, programada utilizando como base el framework Ionic y tomando en cuenta el
trabajo realizado en las librerías PsychoJS. Puede encontrar más información al respecto y
enlaces a los repositorios en el Anexo 2.

Unos contornos explícitos fueron añadidos en ciertos lugares y momentos de la
animación de modo que estos desambiguaran la orientación en la profundidad y rotación de la
figura usando señales de auto-oclusión. Los ejemplos de dichos contornos son apreciados en
la figura 1, donde la adición de un simple contorno de color blanco interno a la silueta
desambigua cuál de las piernas de la bailarina está extendida y por lo tanto determina la
dirección a la que ella está apuntando.

\subsection{Procedimiento}
A cada observador se le suministró un dispositivo
móvil con la aplicación “Spinning Dancer” instalada y a través de dicha aplicación, se expuso
al observador ante un total de 300 repeticiones del estímulo a lo largo de 5 minutos. Durante
este periodo de tiempo fueron presentadas al observador un total de 5 señales transitorias (a
los 26, 76, 148, 200, 258 segundos) que alteraban de forma directa la profundidad de la
imagen con el objetivo de generar un cambio perceptual.

Se le pidió al observador que registrara cualquier tipo de cambio perceptual en la
dirección de la animación, utilizando dos botones ubicados dentro del área táctil del
dispositivo móvil en la parte inferior. Una vez transcurridos los 5 minutos de la prueba se
consultó a cada uno de ellos si habían notado algún tipo de alteración o aparición de alguna
señal durante la animación. Si la aparición de las 5 señales había pasado totalmente
desapercibida para el observador, entonces se procedía a enviar los resultados de su prueba a
un servidor que habíamos dispuesto con anterioridad para recibir y almacenar la información
para su posterior análisis.

\section{Resultados}

Los observadores experimentaron un promedio de 26 cambios perceptuales a lo largo del
experimento (SD = 22,49), con un promedio de latencia entre cambios perceptuales de 24,22
s (SD = 16,62).

Si una señal transitoria es la causante de un cambio perceptual, podríamos naturalmente
esperar encontrar cambios registrados muy cercanos a los tiempos de aparición de dichas
señales, es decir, que la señal produzca un cambio perceptual poco después de su aparición.
Para probar esto, primero calculamos el número de cambios perceptuales ocurridos durante
los 8 segundos siguientes a la aparición de cada señal, moviendo la ventana de 8 segundos
por una fracción de 1 segundo cada vez. Sin embargo, para determinar si el cambio
perceptual fue producido por la señal inconsciente, tuvimos que comparar la frecuencia y
probabilidad de cambios perceptuales cuando una señal fue presentada en contraste a los
periodos donde ninguna señal fue presentada. Por lo tanto también medimos la cantidad de
cambios perceptuales que ocurrieron durante cada fragmento de 8 segundos donde no se
había presentado una señal. Utilizamos una estrategia tecnológica para medir dicho
fragmento de 8 segundos moviéndonos 1 segundo desde el inicio hasta el final del
experimento. Cada fragmento de 8 segundos fue incluido a menos que contuviera en él la
aparición de una señal (en cuyo caso fue contabilizado como mencionamos anteriormente).
Por lo tanto esta estrategia nos proporcionó una base en la que se incluyeron todos los
cambios perceptuales, tanto los estocásticos como los influenciados por las señales.

La información obtenida puede ser apreciada en la figura 2, presentada como la
proporción de cambios perceptuales en relación a la base de cambios (ej, un valor de 2 indica
que ocurrieron el doble de cambios perceptuales en relación con la media de cambios de
base).

Existe una clara tendencia decreciente en esta figura, en la que la mayor parte de los
cambios perceptuales causados por las señales ocurriendo en los primeros segundos luego de
la presentación de dicha señal. Inesperadamente, parece ser que este incremento en los
cambios perceptuales debidos principalmente a la presencia de la señal fue balanceado por un
decrecimiento correspondiente en los cambios perceptuales ocurridos durante la segunda
mitad del fragmento (tanto que incluso descendió por debajo de la media de base). En el
segundo 3 la cantidad de cambios perceptuales no fue tan considerablemente alta en relación
con la base, pero para la mayoría de segundos restantes fue mucho menor que la media de
base, excepto por el segundo 5 donde la diferencia no fue tan notable.

Estos análisis se pueden reducir en términos de la tendencia observada directamente en la
figura 2: Las señales transitorias inconscientes producen un número mayor de cambios
perceptuales (respecto a la base) inmediatamente después de ser presentadas, llegando a
producir incluso el doble en el segundo siguiente a dicha presentación.

\section{Discusión}
La neurociencia es un campo de investigación con un avance vertiginoso y que exige al máximo eficiencia, 
celeridad y automatización para los procesos implicados en la recolección y análisis de datos. 
Si bien hasta el momento ha sabido aprovechar la tecnología existente, es menester que los últimos 
avances matemáticos y en materia tecnológica se conviertan en el brazo fuerte que impulse investigaciones 
como la conectividad funcional del cerebro.

Tecnologías como la inteligencia artificial, machine learning, deep learning y algoritmos de redes 
neuronales proveen herramientas potentes como nunca antes habíamos visto y que permitirían no sólo 
acelerar aún más las investigaciones, sino también, garantizar que los resultados sean muchísimo más 
confiables, replicables y certeros.

La conectividad funcional del cerebro durante la producción de procesos cognitivos y funciones 
mentales superiores es un campo de estudio que sólo se puede abordar en humanos. Esto trae como 
consecuencia ciertas limitaciones a la hora de realizar la investigación, como la falta de pacientes 
dispuestos a participar en procedimientos clínicos estresantes; por lo tanto, la cantidad y calidad 
de los datos se ve comprometida. En este contexto, es esencial optimizar las técnicas estadísticas 
aplicadas para maximizar el potencial de los resultados.

La metodología implementada en la simulación presentada durante este documento fue diseñada específicamente 
para este tipo de situaciones. Además, el hecho de que los resultados obtenidos podrían haber sido mejorados 
es razón suficiente para estimar que con un paradigma experimental más estructurado y con más datos, 
los resultados obtenidos de los pacientes analizados podrían replicarse a más sujetos.

\printbibliography

\end{document}