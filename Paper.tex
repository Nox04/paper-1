\documentclass[jou]{apa6}

\usepackage[spanish]{babel}
\usepackage[utf8]{inputenc}
\usepackage{csquotes}
\usepackage{siunitx}
\usepackage[style=apa,sortcites=true,sorting=nyt,backend=biber]{biblatex}

\DeclareLanguageMapping{spanish}{spanish-apa}
\addbibresource{bibliography.bib}

\title{Inteligencia artificial y psicología. Dos campos de acción, un paradigma en común}
\shorttitle{Conectividad cerebral y machine learning}
\author{Juan D. Angarita y Sharol G. Torres}
\affiliation{Universidad Popular del Cesar}
\date{2018-08-08}

\leftheader{Juan David Angarita, Sharol Giseth Torres}
\rightheader{Inteligencia artificial y conectividad cerebral}

\abstract{En los últimos años, la inteligencia artificial ha desempeñado un papel muy relevante en el intento
de posibilitar la observación, el análisis y la simulación de vías ejecutivas presentes en procesos cognitivos 
como la lectura, sin embargo, la integración de esta tecnología en la investigación psicológica sigue siendo
un paradigma poco explorado en la actualidad. En este trabajo se ha profundizado en una de las posibilidades de 
integración de la inteligencia artificial mediante la creación de una herramienta que permite sintetizar y 
graficar los datos registrados a través de la electrocorticografía. Además, presentaremos ejemplos de 
campos de acción en el estudio de la conectividad cerebral que pueden ser potencializados a través de la 
implementación de algoritmos de inteligencia artificial y ahondaremos en algunos desafíos que todavía quedan 
por resolver antes de que esta tecnología sea implementada de forma extendida en el estudio de procesos 
cognitivos.}

\keywords{inteligencia artificial, conectividad cerebral, electrocorticografía}


\begin{document}
\maketitle
Durante la última década, las investigaciones de la conectividad cerebral han generado una nueva dinámica en el estudio
y el entendimiento del funcionamiento del cerebro humano \parencite{Wen2005}, sin embargo, uno de los desafíos más
importantes dentro de este campo de acción sigue siendo la recolección, procesamiento y análisis de los datos 
requeridos \parencite{doi:10.1146/annurev.psych.56.091103.070311}. De forma paralela, ha surgido la necesidad 
de estudiar las redes temporales que se forman en tiempo real para poder observar cómo las diferentes estructuras 
cerebrales interactúan entre sí \parencite{Swanson2003}.

Una de las propuestas de la neurociencia moderna ha sido el aprovechamiento de los nuevos algoritmos que 
permiten analizar la interdependencia entre señales temporales, además de la emergente teoría de redes complejas 
y la aparición de técnicas como la magnetoencefalografía. \textcite{Sporns2010} ha sugerido 
que la teoría de redes es especialmente adecuada para el estudio de la función cerebral, puesto que nuestro 
cerebro contiene cerca de 1014 sinapsis neuronales y la enorme cantidad de conexiones propicia un entorno ideal 
para la sincronización transitoria o permanente de neuronas dando como resultado la aparición de las funciones 
cognitivas. Por lo tanto, comprender la organización de esta compleja red cerebral, representa 
uno de los desafíos más importantes y emocionantes en el campo de la neurociencia.

La perspectiva del funcionamiento del cerebro ha pasado de la visión frenológica a una visión integradora, 
en la que la función cerebral se lleva a cabo mediante la comunicación entre distintas regiones \parencite{Tononi5033}. 
Es por esto que herramientas como el EEG/MEG resultan de especial utilidad para el estudio de la conectividad
ante estímulos cortos. Dicho estudio se hacía, inicialmente, a partir de las relaciones estadísticas entre las 
series temporales obtenidas por los sensores, sin embargo, a día de hoy, han surgido nuevos métodos de 
análisis como la correlación, la coherencia, la sincronización en fase y la causalidad \parencite{Jirsa2007}; 
y, aunque estos métodos han mejorado procesos vitales como la localización de fuentes, siguen siendo insuficientes 
para segregar y obtener resultados debido al gran número de datos.

Ante la necesidad imperante de ir mucho más allá de mostrar las estadísticas generales de las redes, 
el uso de mecanismos tecnológicos cobra una importancia trascendental. Los algoritmos de inteligencia artificial 
y machine learning son una primera respuesta a dicha necesidad y han probado su utilidad, sobretodo a la hora de 
analizar grandes volúmenes de datos siguiendo patrones bien entrenados.

Una de las tareas que genera mayor demanda de tiempo y recursos es la observación de las combinaciones entre los nodos
y las diferentes frecuencias electromagnéticas. Una forma de solventar este problema es la facilitación de la 
exploración entre nodos \parencite{Massimini2228}: para este fin, hemos desarrollado una herramienta interactiva 
de visualización de redes que permite relacionar medidas estadísticas binarias y ponderadas con frecuencias y 
tiempos para generar espectrogramas y diagramas de coherencia.

\section{Métodos}
\subsection{Participantes}
\subsection{Materiales}
\subsection{Diseño}
\subsection{Procedimiento}

\section{Resultados}

\section{Discusión}

\printbibliography

\end{document}